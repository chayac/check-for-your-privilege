% LaTeX handouts by Will Monroe
% Based on B. E. Burr's CS 109 problem set template

\documentclass[11pt]{article}
\usepackage{newtxtext,newtxmath}

\usepackage{amsmath}
\usepackage{amssymb}
	% packages that allow mathematical formatting
\usepackage{comment}
  % comment out sections

\usepackage{graphicx}
\usepackage{float}
%\usepackage{subfigure} % lisa commented out

\usepackage{tikz}
	% package that allows you to include graphics

\usepackage[hidelinks]{hyperref}
	% and links
\usepackage{algpseudocode}
  % and algorithms
\usepackage{minted}
	% and code

\usepackage[tmargin=1in,bmargin=1in,lmargin=1in,rmargin=1in]{geometry}
\frenchspacing
	% one space after periods
\raggedbottom
	% don't put extra space between sections

\usepackage{fancyhdr}
\pagestyle{fancy}
\renewcommand{\headrulewidth}{0pt}
\renewcommand{\footrulewidth}{0pt} 
\setlength{\headheight}{14.5pt} 
	% allows custom headers

\lhead{}
\rhead{-- \thepage{} --}
\cfoot{}
	% page numbering

\usepackage{titlesec}
\titleformat*{\section}{\large\bfseries}
\titleformat*{\subsection}{\large\itshape\bfseries}
\titleformat*{\subsubsection}{\normalsize\bfseries}
\titleformat*{\paragraph}{\normalsize\bfseries}
\titleformat*{\subparagraph}{\normalsize\bfseries}
\setlength{\parindent}{0cm}
\setlength{\parskip}{3mm plus 3mm minus 1mm}
\titlespacing*{\section}{0pt}{2mm plus 4mm minus 1mm}{-2mm plus 1mm minus 0mm}
\titlespacing*{\subsection}{0pt}{0mm plus 4mm minus 1mm}{-2mm plus 1mm minus 0mm}
\titlespacing*{\subsubsection}{0pt}{0mm plus 4mm minus 1mm}{-2mm plus 1mm minus 0mm}
\titlespacing*{\paragraph}{0pt}{0mm plus 3mm minus 1mm}{1mm plus 1mm minus 0mm}
  % adjust fonts and spacing

\usepackage{framed}
\begin{comment}
\newenvironment{framed}
    {
    \hspace*{-3mm}
    \begin{tabular}{|p{\textwidth}|}
    \hline
    }
    { 
    \\[2mm]\hline
    \end{tabular} 
    \vspace*{-2mm}
    }
  % boxes around paragraphs with titles
\end{comment}

\usepackage{textpos}
\usepackage[super]{nth}
\usepackage{mathtools}
\usepackage{mathdots}

\usepackage{textcomp}
  % \textonehalf for '8 1/2" x 11"'

\usepackage{enumitem}
% \setlist{nolistsep}
  % configure display of enumerations [a), b), c)...]
\usepackage[normalem]{ulem}
  % \sout{} for strikethrough
\usepackage{booktabs}
\usepackage{diagbox}
  % configure display of tables

\newcommand{\vocab}[1]{\textbf{#1}}
\renewcommand{\|}{\mid}

\newcommand{\ra}[1]{\renewcommand{\arraystretch}{#1}}
\newcommand\tab[1][1cm]{\hspace*{#1}}
\newcommand\tabhead[1]{\small\textbf{#1}}

\newcommand{\R}{\mathbb{R}}
\newcommand{\Z}{\mathbb{Z}}
\newcommand{\E}{E}
\newcommand{\Var}{\operatorname{Var}}
\newcommand{\SD}{\operatorname{SD}}
\newcommand{\Cov}{\operatorname{Cov}}
\newcommand{\bigO}{O}

\newcommand{\Uni}{\operatorname{Uni}}
\newcommand{\Ber}{\operatorname{Ber}}
\newcommand{\Bin}{\operatorname{Bin}}
\newcommand{\Geo}{\operatorname{Geo}}
\newcommand{\NegBin}{\operatorname{NegBin}}
\newcommand{\Zipf}{\operatorname{Zipf}}
\newcommand{\HypG}{\operatorname{HypG}}
\newcommand{\Poi}{\operatorname{Poi}}
\newcommand{\Beta}{\operatorname{Beta}}
\newcommand{\N}{\mathcal{N}}
\newcommand{\Gaussian}{\mathcal{N}}
\newcommand{\Exp}{\operatorname{Exp}}

\DeclareMathOperator*{\argmin}{arg\,min}
\DeclareMathOperator*{\argmax}{arg\,max}

\makeatletter
\newcommand{\ack}[1]{\def \@ack {#1}}
\newcommand{\handoutid}[1]{\def \@handoutid {#1}}
\newcommand{\spaceadjust}[1]{\def \@spaceadjust {#1}}
\makeatother

% \ack{}
% \handoutid{}
\spaceadjust{0mm}

\usepackage{multirow}
\usepackage{bm}

% \newenvironment{boxpar}
%     {
%     \hspace*{0mm}
%     \begin{tabular}{|p{0.95\textwidth}|}
%     \hline
%     }
%     {
%     \\[2mm]\hline
%     \end{tabular}
%     \vspace*{0mm}
%     }
  % boxes around paragraphs with titles
  
\usepackage{soul} % for line-break underline


\usepackage{biblatex} %Imports biblatex package
\addbibresource{challenge_refs.bib} %Import the bibliography file
% \usepackage{tocloft}
% \usepackage{minitoc}

% \usepackage{enumitem}
% \setlist{nosep} % or \setlist{noitemsep} to leave space around whole list


\title{CS109 Probability Challenge: "Check For Your Privilege"}
\author{Chaya Carey}
\date{2022-06-04}

\begin{document}

\makeatletter
% the handout title goes here
\begin{textblock*}{0.5\textwidth}(0\textwidth,0mm)
\@author
\end{textblock*}

\begin{textblock*}{0.5\textwidth}(.5\textwidth,0mm)
\hfill
% \@date
June 4, 2022
\end{textblock*}

\begin{center}
\vspace*{2mm}
{\Large \@title} \\
\end{center}
\vskip -9mm
\vskip \@spaceadjust
\rule{\textwidth}{0.5pt}

\vspace*{-4mm}
\hfill {\footnotesize \@ack}
\makeatother


% \let\olditemize=\itemize \let\endolditemize=\enditemize \renewenvironment{itemize}{\olditemize \itemsep0em}{\endolditemize}

% \renewcommand{\baselinestretch}{0.25}\normalsize
% \renewcommand\cftsecafterpnum{\vskip-15pt}
% \renewcommand\cftparskip{0}
% \setlength{\cftbeforesecskip}{0}
% \tableofcontents
% \renewcommand{\baselinestretch}{1.0}\normalsize

The main output of this project is an interactive tool to explore conditional probabilities of income in different groups. The writeup in this document describes the probability theory that underpins the tool, but please go check the tool out for the full picture!

\href{https://observablehq.com/d/b1dfcf5fca53a8b2}{https://observablehq.com/d/b1dfcf5fca53a8b2}

You can also see a demo in the linked video, and the appendices will walk through an example of using the tool.

\section{Code}
Code can be viewed at \href{https://github.com/chayac/check-for-your-privilege}{https://github.com/chayac/check-for-your-privilege}:
\begin{itemize}
    \item \textit{data\_processing\_acs.ipynb}: Jupyter notebook for processing American Community Survey data, running logistic regressions, and exporting summarized data to CSV for Observable
    \item \textit{model/include/model.h}: C++ library for calculating probabilities on a Bayesian network with rejection sampling (not used in the final output)
\end{itemize}

\section{Video}

\newpage

\section{Writeup}

As a software engineer and long-time Seattle resident, I'm acutely aware of how the expansion of "Big Tech"/FAANG companies has impacted the city. These impacts are good, bad, and everything in between, but they are real. I believe that everyone in Big Tech has a responsibility to be aware of their impact and then be a force of good to amplify the positive effects and mitigate the negative ones.

For this project, I developed \textbf{"Check for Your Privilege,"} an interactive data exploration tool to use principles of conditional probability to understand income inequities and identify ways to address them. The name is a tongue-in-cheek reference to the dual purpose: first to "check for your privilege" by quantifying disparities, and then to write a "check for your privilege" to local non-profits working to address these issues.

This writeup will discuss the probability principles that underpin the tool, namely using frequencies to approximate probability, conditional probability, Bayes' theorem, and total probability.

\subsection*{Background}

I have benefited from many of the good things that Big Tech has brought to Seattle, such as access to wider variety of high-income roles, reducing my commute as more companies centering in the city instead of suburbs, and a larger community of like-minded engineers that help me grow. Big Tech's influence has allowed me to stay in Seattle, instead of having to relocate to the Bay Area like I did earlier in my career.

But I also recognize that the impact of Big Tech is both good and bad, and both the positive and negative impacts are not distributed equitably. In the excellent book \textit{The Gentrification of the Internet: How to Reclaim Our Digital Freedom} (highly recommended!), the author cites Northern California as an example of how Big Tech means that "big cities are getting \textbf{more unequal and less diverse}."\cite{gentrification} The two main mechanisms for this change are:
\begin{itemize}
    \item Urban gentrification in areas where Big Tech companies are headquartered
    \item Lack of diversity among Big Tech employees, which is "disproportionately White, male, and young"\cite{gentrification}
\end{itemize}

While gentrification has both positive and negative impacts, "gentrification makes inequality more visible [...] benefits aren't even distributed. Urban gentrification tends to make rich people richer and poor people poorer." In Seattle, we see this as homelessness and poverty increase yet the average household income has risen over \$100,000.

To investigate these issues, I am analyzing data on the distribution of high and low income across Seattle residents. All data is from the 2019 American Community Survey's Public Use Microdata Sample for Seattle.\cite{acs}\cite{pums} 

\subsection*{Visually Comparing Distributions of Income Between Groups}

Intuitively, I think of disparities as differences in outcomes based on unrelated demographic factors. For example, if no racial disparities existed in income, then a BIPOC person would be just as likely as a white person to have a certain income level.

To explore this intuition, I used the frequentist definition of probability\cite{piech:prob} to estimate the probability of different combinations of demographic factors by counting the number of corresponding events in the dataset. The \href{https://observablehq.com/d/b1dfcf5fca53a8b2#cell-1196}{interactive tool} lets you compare the counts of people with high and low income between two different groups, which are the count of people that are part of the given group \textbf{and} have the given income level.\cite{piech:and}

The \href{https://observablehq.com/d/b1dfcf5fca53a8b2#cell-1196}{interactive graphs} are normalized to the size of the group to visually approximate the percentage of people in a group at each income level. If the two groups were equally likely to have a certain income level, the two distributions would be visually similar.

For example, selecting the characteristics that are overrepresented in tech (being white, male, and young) shows a comparison of the income distributions of non-BIPOC males under 35 vs. BIPOC non-males over 35.  Visually, we can see that BIPOC non-males over 35 are less likely to be high income but also less likely to be low income. But if you also condition on tech workers, BIPOC non-males over 35 who aren't tech workers are more likely to be low income.

\subsection*{Formalizing Income Distributions with Conditional Probability}

Visually we can see that each group has a different percentage of people at each income level. To formalize this, we can use conditional probability\cite{piech:cond} to estimate the probability of income level within a group. Using the frequentist definition, we can approximate the conditional probability of having an income level within a group by counting the number of people who have that income level within the group and dividing it by the number of people in the group. For example, the probability of high income among BIPOC people is P(High Income | BIPOC), or approximately N(High Income and BIPOC)/N(BIPOC).

\subsection*{Defining Fairness with Conditional Probability}

How do we know what is equitable? A useful framework is fairness in AI, which analyzes the bias of an algorithm. A biased algorithm "systematically and unfairly" discriminates against certain groups, which can result in "quality of service" harms and allocation harms where the algorithm doesn't work as well for different groups of people and can result in unequal distribution of outcomes.\cite{creel}

We can find an analogue in income distribution by analyzing the bias associated with how income levels are distributed for different types of people. To quantify this, I looked to the "parity" definition of fairness through awareness. Under the parity definition of fairness, an algorithm is fair if "the probability that the algorithm makes a positive prediction is the same regardless of being conditioned on demographic variable".\cite{piech:fair}

This definition is based on the equality of conditional probability between groups, which we can use here by calculating the conditional probability of an outcome for each group. For example:

$$P(\text{High Income | White}) = P(\text{High Income | Non-White})$$

In the \href{https://observablehq.com/d/b1dfcf5fca53a8b2#cell-1189}{interactive tool}, you can compare the conditional odds of having high or low income conditioned on the demographic factors you select, which show  how much more or less likely the outcome is given the condition.\cite[p.~70]{ross}

\subsection*{Decomposing Conditional Probability with Bayes' Theorem}

Now that we have defined equity in terms of conditional probability, we can decompose the conditional probability using Bayes' theorem.\cite{piech:bayes} 

For example, the probability of having high income for a BIPOC person P(H|F) is a combination of: 
* P(F): the probability of being BIPOC
* P(H): the probability of having high income
* P(F|H): the probability of being BIPOC given high income

The \href{https://observablehq.com/d/b1dfcf5fca53a8b2#cell-373}{interactive tool} breaks down the calculation of each component using the approximate conditional probabilities from the data.

\subsection*{Donating Effectively Using Bayes' Theorem}

This is the fun part! Now we get to use all these probability tools to determine how donating money, time, skills, etc. to different organizations can impact the income disparities we've seen. One common tool in philanthropy is theory of change, which requires that you set a goal \textbf{and} identify specific pathways for achieving that goal."\cite{effphil}\cite{theorychange} 

I propose using Bayes' theorem to identify the mechanisms for achieving a goal. For example, say that you want to reduce the probability of being low income among BIPOC people. Bayes' theorem decomposes this conditional probability, showing three potential avenues for accomplishing the goal:
\begin{itemize}
    \item Reducing the probability of being low income among BIPOC people (the posterior probability)
    \item Reducing the overall probability of being low income (the prior probability)
    \item Reducing the probability of being BIPOC among low income people (the likelihood)
\end{itemize}

Reducing probabilities may seem abstract, but by looping back to the beginning and approximating these probabilities with frequencies. Then we see that these probabilities can be impacted by changing the counts in groups:
\begin{itemize}
    \item All people who are low income
    \item People who are low income and BIPOC
    \item People who are not low income
    \item People who are BIPOC but not low income
\end{itemize}
in the last section, we broke down these probabilities into estimates using the frequencies observed in the sample data. All of these probabilities are determined by the counts of different groups of people:
* All people who are low income
* People who are low income and BIPOC
* People who are not low income
* People who are BIPOC but not low income

One final probability tool to consider is the law of total probability. The fourth component of Bayes' theorem, the normalization factor, may seem difficult to impact for demographic features. Increasing the BIPOC population of Seattle might be a hard goal, but the law of total probability lets us again decompose this probability into multiple parts:

$$P(F) = P(F \| H) P(H) + P(F \| H^C) P(H^C)$$

The law of general probability allows us to expand even beyond the factors in the model, such as the probability of STEM degrees among BIPOC people. 

Putting these pieces together, we see that  supporting positive change in the BIPOC community in any way will have some impact on the probability of BIPOC people having low income. Will it have the largest possible impact on that specific probability? Probably not, but it will also have some impact on other outcomes too, and maybe you want to take a more general approach. It's your theory of change, and it's your money (or time, or attention, or whatever you have to give)! Perhaps the best lesson from Bayes' theorem is the simplest one: we are all dealing with uncertainty so we're not expecting a perfect answer - just a better one, now that we've learned more.

\newpage 

\appendix
\addcontentsline{toc}{section}{Appendices}
\section*{Appendices}
\section{Sources}


% \printbibliography %Prints bibliography
\printbibliography[heading=none]


\section{Derivations}

In my project, I use the American Community Survey data to approximate probabilities. This section covers the derivations of the conversions among the frequentist probability estimate, conditional probability, and Bayes' theorem.

$H$ = high income

$F$ = factors

$G$ = some other set of factors

$n$ = total number of people in dataset

Bayes theorem:
\begin{align*}
    P(H\|F) &= \dfrac{P(F \| H) P(H)}{P(F)} && \text{Bayes' theorem} \\
    P(F|H) &= \frac{P(FH)}{P(H)} && \text{Def of Cond Prob} \\
    &\approx \frac{
    	(\text{$n(F$ and $H$})) / n
    }{
    	(\text{$n(H)$}) / n
    } && \text{Def of Prob} \\
    &\approx \frac{n(FH)}{n(H)} \\
    P(H) &\approx \frac{n(H)}{n} \\
    P(F) &\approx \frac{n(F)}{n} \\
    P(H\|F) &= \frac{n(FH)/n(H) \cdot n(H)/n}{n(F)/n} && \text{Plug back into Bayes' theorem} \\
    &= \frac{n(FH)}{n(H)} \cdot \frac{n(H)}{n} \cdot \frac{n}{n(F)}  \\
    &= \frac{n(FH)}{n(F)}
\end{align*}

Odds:
\begin{align*}
    O(H | F) &= \frac{P(H | F)}{P(H^C | F)} && \text{Def of odds} \\
    &= \frac{n(FH)/n(F)}{n(FH^C)/n(F)} \\
    &= \frac{n(FH)}{n(FH^C)} \\
    &= \frac{n(FH)}{n - n(FH)} \\
    O(H | F) &= O(G | F) \\
    \frac{n(FH)}{n - n(FH)} &= \frac{n(GH)}{n - n(GH)} \\
    \frac{O(H | F)}{O(G | F)} &= \frac{n(FH)/(n - n(FH))}{n(GH)/(n - n(GH))}
\end{align*}

Fairness:
\begin{align*}
    P(H | F) &= P(H | F^C) && \text{Def of parity}\\
    \frac{n(FH)}{n(H)} &= \frac{n(F^CH)}{n(H)} \\
    n(FH) &= n(F^CH) \\
\end{align*}

Law of total probability:
\begin{align*}
    P(F) &= P(F \| H) P(H) + P(F \| H^C) P(H^C) \\
    P(H | F) &= \dfrac{P(F \| H) P(H)}{P(F \| H) P(H) + P(F \| H^C) P(H^C)} \\
    &= \dfrac{n(FH)/n(H) \cdot n(H)/n}{n(FH)/n(H) \cdot n(H)/n + n(FH^C)/n(H^C) \cdot n(H^C)/n}
\end{align*}

\end{document}